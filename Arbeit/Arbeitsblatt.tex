\documentclass[10pt,a4paper]{article}
\usepackage[utf8]{inputenc}
\usepackage{amsmath}
\usepackage{amsfonts}
\usepackage{amssymb}
\begin{document}
\section*{Ebola-Fieber}
Das Ebola-Fieber wird durch einen Virus ausgelöst. Zunächst haben Patienten sehr unspezifische Symptome, wie Fieber und Durchfall. Jedoch werden feine Blutgefäße durch die Infektion zerstört, wodurch es zu Nieren- und Leberschäden kommen kann, sowie inneren Blutungen, die sogar zu Kreislaufversagen und Tod führen können.\\
Jeder, der mit Ebola infiziert ist, kann etwa 2,18 weitere Personen anstecken. Von den Infizierten sterben etwa 50\% an der Krankheit.\\
\subsection*{Aufgabe 1:}
Fertige ein Modell an, mit welchem man den Verlauf einer Ebola-Epedemie in Guinea, Liberia, Sierra Leone und Nigeria nachstellen kann. In diesen Ländern leben 11,5 Millionen Menschen und es gibt 4108 bestätigte Fälle von Ebola-Fieber.\\
Vergleiche dein Modell mit dem deines Nachbarn, entscheidet euch für ein Modell und versucht es in einem Tabellenkalkulationsprogramm umzusetzen.\\

Wir nehmen an, dass Personen die einmal erkrankt sind, nicht mehr erkranken. Zusätzlich lassen sich 70\% der Infizierten in einem Krankenhaus behandeln, wodurch die Überlebenswahrscheinlichkeit auf 60\% steigt.\\
\subsection*{Aufgabe 2:}
Erweitert euer Modell mit Hilfe der neuen Informationen und setzt auch das neue Modell mit einem Tabellenkalkulationsprogramm um.\\

Auch Tote sind noch ansteckend. Das Virus befindet sich in Körperflüssigkeiten und kann durch die Schleimhäute gesunder Personen aufgenommen werden. Da in Afrika bestimmte Beerdigungsrituale durchgeführt werden, ist es wahrscheinlich, sich bei diesen Gelegenheiten anzustecken. Wir gehen jedoch davon aus, dass diejenigen, die im Krankenhaus gestorben sind, ohne Ansteckungsgefahr für andere beerdigt werden.\\
\subsection*{Aufgabe 3:}
Erweitert euer Modell mit Hilfe der neuen Informationen und setzt auch das neue Modell mit einem Tabellenkalkulationsprogramm um. 
\end{document}