\documentclass[10pt,a4paper]{article}
\usepackage{pdflscape}
\usepackage[utf8]{inputenc}
\usepackage{amsmath}
\usepackage{amsfonts}
\usepackage{amssymb}
\usepackage{tabularx}%Fuer eigene Tabulararten
\usepackage{longtable}
\newcolumntype{L}[1]{>{\raggedright\arraybackslash}p{#1}} % linksbündig mit Breitenangabe
\newcolumntype{C}[1]{>{\centering\arraybackslash}p{#1}} % zentriert mit Breitenangabe
\newcolumntype{R}[1]{>{\raggedleft\arraybackslash}p{#1}}
\begin{document}
\begin{landscape}
\noindent
\begin{tabular}{L{0.4\textwidth}|C{1\textwidth}|R{0.6\textwidth}}
Phase & Arbeitsauftrag & Sozialform\\
\hline
Begrüßung & keine & Lehrervortrag\\
\hline
Einstieg & Lesen der Informationen über Ebola & Einzelarbeit\\
\hline
1. Erarbeitungsphase (a) & Modell entwickeln zur Simulation einer Ebola Epidemie & Einzelarbeit\\
\hline
1. Erarbeitungsphase (b) & Modell mit dem Nachbarn vergleichen, ein gemeinsames Modell erarbeiten & Partnerarbeit\\
\hline
1. Erarbeitungsphase (c) & Das gemeinsame Modell mit Hilfe von Tabellenkalkulationsprogramm simulieren & Partnerarbeit\\
\hline
1. Reflexionsphase & Modelle der SuS werden vorgestellt und Stärken und Schwächen der Modelle besprochen & Schülervorträge + Diskussion\\
\hline
2. Erarbeitungsphase & Mit neuen Informationen und den Überlegungen aus der Diskussion wird das Modell verbessert und wieder eine Simulation durchgeführt & Partnerarbeit\\
\hline


\end{tabular}
\end{landscape}
\end{document}