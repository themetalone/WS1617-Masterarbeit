\section{Einleitung}
\ellen
Innerhalb dieser Arbeit soll, mit Hilfe von mathematischen Modellen und Simulationen, untersucht werden, wie sich Epidemien ausbreiten und wie diesen sinnvoll begegnet werden kann.\\
Dazu werden zunächst Modelle vorgestellt, die helfen sollen, den Verlauf von Epidemien zu beschreiben.\\ 
Dabei beschreibt Ellen Holzer Single Population Ansätze, während Steffen Holzer diese Ansätze auf mehrere Populationen erweitert.\\ 
Diese werden dann in einer Simulation verwendet, um zu testen, welchen Verlauf eine Epidemie nehmen würde, falls nichts zu ihrer Eindämmung unternommen wird. Danach wird simuliert, wie sich verschiedene Strategien auf den Verlauf der Epidemie auswirken und mit welchen Kosten das verbunden ist.\\
Die Simulation wird von Steffen Holzer programmiert und durchgeführt. Die Auswertung der Daten geschieht dann durch Ellen Holzer.\\
Zuletzt wird ein Schulprojekt geplant, in dem Schülerinnen und Schülern die mathematischen Grundlagen zur Durchführung einer solchen Untersuchung vermittelt werden sollen. Ebenso sollen sie lernen eine solche Untersuchung selbstständig durchzuführen.\\
 Dabei übernimmt Ellen Holzer die ersten Stunden, in deren Verlauf der Modellierungskreislauf erschlossen und durchlaufen wird. Steffen Holzer übernimmt den Abschluss der Reihe, bei dem die Erweiterung des Modells auf mehrere Populationen und die Evaluation des Modelles im Vordergrund stehen.  