\section{Didaktische Umsetzung}
\ellen
Mit dieser Unterrichtsreihe sollen vor allem zwei Dinge aus dem Lehrplan umgesetzt werden. Zum einen soll der Computer den Mathematikunterricht unterstützen und bereichern. Erst durch den Computereinsatz entsteht die Möglichkeit, realitätsnahe Modellierungen durchzuführen und auszuwerten. \\
Zusätzlich sollen die SuS den Prozess der mathematischen Modellbildung lernen. Die Betrachtung der Krankheitsausbreitung bietet sich dafür in besonderer Weise an. Zunächst ist die Motivation sehr hoch, da die SuS bei vielen Krankheiten, wie Grippe, selbst betroffen sind. Zudem lässt sich der Kreislauf der Modellbildung sehr gut nachvollziehen, da die SuS immer wieder feststellen werden, dass sie bestimmte Aspekte nicht ausreichend berücksichtigt haben und ihr Modell die Realität noch nicht zufriedenstellend beschreibt.\\
Um die Motivation zu erhalten sollte mit dem einfachsten Modell gestartet werden. Daher wird zunächst der Single Population Ansatz untersucht.
\subsection{Single Population Ansätze}
\ellen
Die Unterrichtsreihe sollte wie der Modellierungskreislauf aufgebaut werden. Die SuS sollten also zunächst über Ebola informiert werden. Danach können sie selbstständig verschiedene Modelle erstellen, anhand derer der Verlauf von Ebola rekonstruiert werden kann. Mit geschickter Auswahl der Vorbereitungsliteratur kann man jedoch dafür sorgen, dass einige SuS auf ein Modell kommen, dass dem SIR-Modell sehr nahe kommt. Falls jedoch ein Modell entsteht, dass besser ist, sollte die Lehrkraft in der Lage sein, darauf zu reagieren und diese Modell dann zu verwenden.\\
Die SuS können mit ihrem Modell und mit der Hilfe eines Tabellenkalkulationsprogramms einen Krankheitsverlauf für eine kleine Population simulieren. Dabei sollte ihnen auffallen, dass das Modell noch schwächen hat. Beispielsweise wir nicht zwischen toten und immunen Individuen unterschieden. Zudem hat jeder der erkrankt die selben Chancen geheilt zu werden, egal ob dieser behandelt wird oder nicht.\\
 Durch diese Erkenntnisse, sind die SuS in der Lage das Modell zu verbessern. Somit durchlaufen sie schon mehrere Male den Modellierungskreislauf.\\
 Um zuletzt auf das SEIHFRD-Modell zu kommen brauchen die SuS wahrscheinlich die Hilfe der Lehrkraft, denn die Einführung der Klasse \glqq E\grqq{} könnte Probleme bereiten.\\
 Diese Klasse kann jedoch durch die Lehrkraft, mit Verweis auf die Methoden zur Eindämmung der Krankheitsausbreitung, eingeführt werden. So machen beispielsweise Quarantänen nur dann Sinn, wenn es die Klasse \glqq E\grqq{} gibt, da sonst nicht zwischen denen unterschieden werden kann, die bereits ansteckend sind, aber selbst noch keine Symptome zeigen und jenen, bei denen die Krankheit bereits vollständig ausgebrochen ist.  
\subsection{Netzwerkansätze}