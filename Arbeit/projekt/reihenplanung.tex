\begin{landscape}
\subsection{Reihenplanung}
\noindent
\begin{longtable}{|C{0.05\textwidth}|L{0.2\textwidth}|L{0.3\textwidth}|L{0.25\textwidth}|L{0.3\textwidth}|L{0.25\textwidth}|L{0.1\textwidth}|}
\hline
Std&Inhalte&Grobziele \& Kompetenzen&Wdh \& Festigung&didakt. \& method. Auswahl&Medien&Sonstiges\\
\hline\hline
\endhead
\hline
\endfoot
1\&{}2 &\emph{SIHDR}&\begin{itemize}
	\item Schüler durchlaufen den Modellierungskreislauf zwei mal
	\item K2
	\item K3
\end{itemize}&--&\begin{itemize}
	\item Modell von \emph{SID} zu \emph{SIHDR}
	\item TPS zur Modellbildung
	\item Zwischenergebnisse werden in der Klasse besprochen
	\item Durch Aufgabenstellung wenig Varianz in Modellierung
\end{itemize}&\begin{itemize}
	\item Ta\-bel\-len\-kal\-ku\-la\-tion
\end{itemize}&--\\\hline
3\&{}4&\begin{itemize}
	\item Mo\-del\-lier\-ungs\-kreis\-lauf
	\item \emph{SIHFDR}
\end{itemize}&b&c&d&e&f\\\hline
5\&{}6&\begin{itemize}
	\item Maß\-nahmen gegen Ausbreitung
	\item Modell\-ierungs\-kreislauf
	\item Erwei\-terung auf Pandemien
\end{itemize}&\begin{itemize}
	\item Die Schüler entwickeln Maßnahmen zur Eindämmung und erweitern das bisherige Modell auf mehrere Populationen
	\item K2
	\item K3
\end{itemize}&\begin{itemize}
	\item Modell der Vorstunde
	\item Zu\-sam\-men\-hang Netzwerksmodell und Iterationsvorschriften
	\item Modellierungs\-kreislauf
\end{itemize}&\begin{itemize}
	\item Modellierungs\-kreislauf durch Anwendung
\end{itemize}&\begin{itemize}
	\item Verwendung des Schülermodells
	\item Hand\-lungs\-o\-ri\-en\-tiert
	\item Teils mit TK
\end{itemize}&--\\
\end{longtable}
\end{landscape}