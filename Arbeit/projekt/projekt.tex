\section{Durchführung als Projekt}
\subsection{Ziele}
\steffen
Ziel der Unterrichtsreihe ist es, die Schüler den mathematischen Modellierungskreislauf herleiten und durchführen zu lassen. Als Modellierungsobjekt dient dabei die Modellierung von Epidemien und Pandemien. 


\begin{landscape}
\subsection{Reihenplanung}
\noindent
\begin{longtable}{|C{0.05\textwidth}|L{0.2\textwidth}|L{0.3\textwidth}|L{0.25\textwidth}|L{0.3\textwidth}|L{0.25\textwidth}|L{0.1\textwidth}|}
\hline
Std&Inhalte&Grobziele \& Kompetenzen&Wdh \& Festigung&didakt. \& method. Auswahl&Medien&Sonstiges\\
\hline\hline
\endhead
\hline
\endfoot
1\&{}2 &\emph{SIHDR}&\begin{itemize}
	\item Schüler durchlaufen den Modellierungskreislauf zwei mal
	\item K2
	\item K3
\end{itemize}&--&\begin{itemize}
	\item Modell von \emph{SID} zu \emph{SIHDR}
	\item TPS zur Modellbildung
	\item Zwischenergebnisse werden in der Klasse besprochen
	\item Durch Aufgabenstellung wenig Varianz in Modellierung
\end{itemize}&\begin{itemize}
	\item Ta\-bel\-len\-kal\-ku\-la\-tion
\end{itemize}&--\\\hline
3\&{}4&\begin{itemize}
	\item Mo\-del\-lier\-ungs\-kreis\-lauf
	\item \emph{SIHFDR}
\end{itemize}&b&c&d&e&f\\\hline
5\&{}6&\begin{itemize}
	\item Maß\-nahmen gegen Ausbreitung
	\item Modell\-ierungs\-kreislauf
\end{itemize}&\begin{itemize}
	\item Die Schüler entwickeln Maßnahmen zur Eindämmung
	\item K2
	\item K3
\end{itemize}&\begin{itemize}
	\item Modell der Vorstunde
	\item Zu\-sam\-men\-hang Netzwerksmodell und Iterationsvorschriften
	\item Modellierungs\-kreislauf
\end{itemize}&\begin{itemize}
	\item Modellierungs\-kreislauf durch Anwendung
\end{itemize}&\begin{itemize}
	\item Verwendung des Schülermodells
	\item Hand\-lungs\-o\-ri\-en\-tiert
	\item Teils mit TK
\end{itemize}&--\\
\end{longtable}
\end{landscape}
\subsection{Projektstunde 1 \& 2}\ellen
\subsubsection{Bemerkungen zur Lerngruppe}
Bei der Planung der ersten beiden Stunden war uns lediglich bekannt, dass es sich um Schülerinnen und Schüler der zehnten Klasse eines Gymnasiums für Hochbegabte handelt\\
Zudem seien die Grundlagen der Schülerinnen und Schüler höchst unterschiedlich. Der Umgang mit Tabellenkalkulationsprogrammen sollte jedoch in Grundzügen beherrscht werden. 
\subsubsection{Methodische Überlegungen}
Um die Motivation zu fördern und eine möglichst hohe Nachhaltigkeit zu erhalten, sollen die Unterrichtsstunden handlungsorientiert ausgerichtet werden. Dazu sollen die Schülerinnen und Schüler zunächst alleine Rechenoperationen formulieren, mit deren Hilfe die Größe der Klassen (Gesund, Krank und Tod) berechnet werden können.\\
Die Schülerinnen und Schüler haben bereits vorher im Mathematikunterricht gelernt, wie sie aus Texten Terme aufstellt. Dadurch können die Schülerinnen und Schüler früh Erfolge erringen um dadurch die Motivation zu behalten und um ihnen die Angst vor dem Simulieren nehmen.\\
Durch die Einzelarbeit ist gewährleistet, dass alle Schülerinnen und Schüler sich am Unterricht beteiligen. Sobald alle Schülerinnen und Schüler ihre Terme aufgestellt haben, vergleichen sie ihre mit denen ihres Nachbarn.\\
Dadurch können die Schülerinnen und Schüler mathematisches argumentieren einüben. Zusätzlich entsteht eine gewisse Sicherheit bezüglich des eigenen Vorgehens, dadurch kann die Hemmschwelle gesenkt werden, die eigenen Ergebnisse der Klasse vorzustellen.\\
Nachdem sich alle Gruppen auf Vorschriften geeinigt haben, werden diese der Klasse vorgestellt.\\
In der Klasse werden die Vorschriften vorgestellt. Dabei sollten falsche Vorschriften nicht direkt durch die Lehrperson korrigiert werden. Die Schülerinnen und Schüler können später über die Gleichungen diskutieren und gemeinsam eine korrekte Lösung erarbeiten.\\
Mit diesen Vorschriften sollen die Schülerinnen und Schüler dann eine kleine Simulation mit Hilfe von Tabellenkalkulationsprogrammen erstellen.\\
Der Umgang mit solchen Programmen sollte in Grundzügen bereits beherrscht werden und wird an dieser Stelle vertieft.\\
Nachdem diese Programme angefertigt und in der Klasse besprochen wurden, sollen die Schülerinnen und Schüler Vorschläge zur Verbesserung des Modells machen. Manche dieser Vorschläge werden dann in einem zweiten Programm umgesetzt.\\
Dadurch haben die Schülerinnen und Schüler den Modellierungskreislauf durchgeführt. 
\documentclass[10pt,a4paper]{article}
\usepackage{pdflscape}
\usepackage[utf8]{inputenc}
\usepackage{amsmath}
\usepackage{amsfonts}
\usepackage{amssymb}
\usepackage{tabularx}%Fuer eigene Tabulararten
\usepackage{longtable}
\newcolumntype{L}[1]{>{\raggedright\arraybackslash}p{#1}} % linksbündig mit Breitenangabe
\newcolumntype{C}[1]{>{\centering\arraybackslash}p{#1}} % zentriert mit Breitenangabe
\newcolumntype{R}[1]{>{\raggedleft\arraybackslash}p{#1}}
\begin{document}
\begin{landscape}
\noindent
\begin{tabular}{L{0.4\textwidth}|C{1\textwidth}|R{0.6\textwidth}}
Phase & Arbeitsauftrag & Sozialform\\
\hline
Begrüßung & keine & Lehrervortrag\\
\hline
Einstieg & Lesen der Informationen über Ebola & Einzelarbeit\\
\hline
1. Erarbeitungsphase (a) & Modell entwickeln zur Simulation einer Ebola Epidemie & Einzelarbeit\\
\hline
1. Erarbeitungsphase (b) & Modell mit dem Nachbarn vergleichen, ein gemeinsames Modell erarbeiten & Partnerarbeit\\
\hline
1. Erarbeitungsphase (c) & Das gemeinsame Modell mit Hilfe von Tabellenkalkulationsprogramm simulieren & Partnerarbeit\\
\hline
1. Reflexionsphase & Modelle der SuS werden vorgestellt und Stärken und Schwächen der Modelle besprochen & Schülervorträge + Diskussion\\
\hline
2. Erarbeitungsphase & Mit neuen Informationen und den Überlegungen aus der Diskussion wird das Modell verbessert und wieder eine Simulation durchgeführt & Partnerarbeit\\
\hline


\end{tabular}
\end{landscape}
\end{document}
\subsubsection{Materialien}
\noindent\frame{\includegraphics[width=\textwidth]{projekt/Arbeitsblatt_1}}
\subsubsection{Erwartungshorizont}
Da die Schülerinnen und Schüler bereits gelernt haben, wie sie Gleichungen aufstellen wird erwartet, dass sie zunächst einige Schritte ausprobieren und danach eine Vorschrift für eine beliebige Zeit angeben. Dies könnte dann etwa so aussehen:\\
Zunächst sollten die Kranken betrachtet werden, da diese lediglich von der Anzahl der Kranken zuvor abhängt.\\
$ K(1) = K$\\
$ K(2) = 2,18 \cdot K(1) - K(1) = 1,18 \cdot K(1)$\\
$ K(3) = 1,18 \cdot K(2) = 1,18^2 \cdot K(1)$\\
$ K(t) = 1,18^{t-1} \cdot K(1)$\\

Danach können sie eine Vorschrift für die Gesunden mit der gleichen Vorgehensweise finden.\\
$ G(1) = G$\\
$ G(2) = G(1) - 2,18 \cdot K(1) + 0,5 \cdot K(1) = G(1)- 1,68 \cdot K(1)$\\
$ G(3) = G(2) - 1,68 \cdot K(2)$\\
$ G(t) = G(t-1) - 1,68 \cdot K(t-1)$\\
Diese Gleichung kann selbstverständlich auch anders formuliert werden und explizit dargestellt werden. Diese Darstellung lässt sich jedoch leicht mit Tabellenkalkulationsprogrammen durch ziehen der Formeln umsetzen.\\

Ebenso sollen die Schülerinnen und Schüler Vorschriften für die Anzahl der Toten angeben.\\
$ T(1) = 0 $\\
$ T(t) = T(t-1) + 0,5 \cdot K(t-1)$\\


\noindent\frame{\includegraphics[width=\textwidth]{projekt/erwartung_gtk}}\newpage
\noindent\frame{\includegraphics[width=\textwidth]{projekt/erwartung_gtih}}\newpage
\noindent\frame{\includegraphics[width=\textwidth]{projekt/erwartung_gktihb}}\newpage
\subsubsection{Schülerprodukte}
\noindent\frame{\includegraphics[width=\textwidth]{projekt/leistung_1_1}}\newpage
\noindent\frame{\includegraphics[width=\textwidth]{projekt/leistung_1_2}}\newpage
Alle Produkte der Schülergruppen entsprechen mindestens dem Erwartungshorizont. Das oben angeführte Beispiel zeigt aber zusätzlich ein Schülerprodukt, das die Erwartungen weit übertrifft. Darin zeigt sich, dass eine Differenzierung der Aufgabenstellungen dringend notwendig ist. Einige Gruppen benötigten viel Zeit um sich mit dem Tabellenkalkulationsprogramm vertraut zu machen, während das obige Beispiel zeigt, dass es auch Gruppen gab die nach Bearbeitung der Aufgabe noch die Zeit hatten, einen Graphen zu plotten der den Verlauf nochmal verdeutlicht.  

\subsubsection{Reflexion der Stunde}
Die Schülerinnen und Schüler haben zum Teil neue Funktionen der Tabellenkalkulationsprogramme kennen gelernt. Zudem haben sie Grundlegende Vorgänge der Modellbildung genutzt und eigenständig Modelle aufgestellt. In diesem Zusammenhang wurde auch das Aufstellen von Funktionen wiederholt und vertieft.\\
Die Schülerbeteiligung war sehr hoch, ohne Anstoß durch die Lehrperson haben sie die Vorschläge der Gruppen gemeinsam korrigiert und in einem angenehmen Klima fachlich diskutiert. Zusätzlich wurden schon früh Verbesserungsvorschläge für das Modell angebracht.\\
Da die Klasse unbekannt ist, war die Zusammensetzung in der Partnerarbeit nicht optimal. Wenn man die Lerngruppe besser kennt, könnte man Teams bilden, die sich gegenseitig unterstützen. Dadurch könnten die großen Unterschiede im Zeitbedarf angepasst werden.\\
Ist dies jedoch nicht möglich, sollte es Aufgaben geben, die die besonders schnellen zusätzlich erledigen können.\\
Statt der beiden Simulationen konnten die Gruppen in der Stunde nur eine Simulation durchführen. Dies ist jedoch von Vorteil für die nächste Sitzung, da an diesem Tag acht von siebzehn Schülern fehlten.\\
Daher wird in der nächsten Stunde eine größere Wiederholung stattfinden. Von allen Gruppen wird dann lediglich die zweite Simulation durchgeführt und die dritte Simulation steht für die besonders schnellen Schülerinnen und Schüler zur Verfügung. 
\newpage
\subsection{Projektstunde 3 \& 4}\ellen
\subsubsection{Bemerkungen zur Lerngruppe}\label{ssec:project:34:lerngruppe}
\subsubsection{Methodische Überlegungen}
\subsubsection{Verlaufsskizze}
\subsubsection{Materialien}
\subsubsection{Erwartungshorizont}
\subsubsection{Schülerprodukte}
\subsubsection{Reflexion der Stunde}
\newpage
\subsection{Projektstunde 5 \& 6}\steffen
\subsubsection{Bemerkungen zur Lerngruppe}
Ergänzend zu den Einschätzungen in Abschnitt \ref{ssec:project:34:lerngruppe} zeigt sich die Lerngruppe als durchaus motiviert. Die Schülerinnen und Schüler, die in der 1. und 2. Stunde wegen innerschulischer Aktivitäten nicht am Unterricht teilgenommen haben, gliedern sich gut in das Leistungsspektrum der Klasse ein. Es besteht weiterhin das für Informatikkurse charakteristische ``Leistungstal'' in der Mitte des Leistungsspektrums. Da es sich um eine Hochbegabtenklasse handelt, entspricht das untere Ende des Leistungsspektrums etwa einer durchschnittlichen Leistung in Regelschulen. Die Schülerinnen und Schüler sind interessiert am Kontext der Reihe, fordern aber auch Beschäftigung ein. Die Reaktion auf fragend-entwickelnden Unterricht lässt sich als zäh beschreiben. 

Das Vorwissen in der Klasse ist stark heterogen. Obwohl viele der Schüler ein Zertifikat über den Einsatz von Excel verfügen, ist der aktive Einsatz aber bei den meisten einige Jahre her. Dementsprechend muss beim Einsatz von Tabellenkalkulation entsprechend unterstützt werden. Informatische Grundbildung ist aber breit in der Klasse vorhanden. Begriffe wie ``Variable'' und ``Anweisung'' werden verstanden. 

Schüler R\footnote{Name wegen Datenschutz anonymisiert} hebt sich weiterhin von der restlichen Klasse ab im Hinblick auf die Leistung deutlich nach oben ab. Bezüglich der Sozialkompetenz zeigt sich aber Nachholbedarf. R fordert auch explizit mehr Handlungsspielräume im Unterricht.

Eine Gruppe von fünf Schülerinnen erfordert eine etwas stärkere Lehrerpräsenz während der Bearbeitung von Aufgaben, da sonst deren Aufmerksamkeit sich anderen, unterrichtsirrelevanten Themen zuwendet. 

Eine weitere Gruppe von drei Schülern benötigen ebenfalls mehr Lehrerpräsenz, da diese Gruppe zu alternativen Lösungsansätzen neigt, die jedoch in den vergangen Stunden nicht immer zu einer Lösung führt.
\subsubsection{Methodische Überlegungen}
\begin{landscape}
\subsubsection{Verlaufsskizze}
\noindent
\begin{longtable}{|C{0.3\textwidth}|L{0.8\textwidth}|L{0.4\textwidth}|}
\hline
Phase & Arbeitsauftrag & Sozialform\\
\hline\hline
\endhead
\hline
\endfoot

\end{longtable}
\end{landscape}
\subsubsection{Materialien}
\subsubsection{Erwartungshorizont}
\subsubsection{Schülerprodukte}
\subsubsection{Reflexion der Stunde}