\documentclass[11pt]{beamer}
\usetheme{Bergen}
\usepackage[utf8]{inputenc}
\usepackage{amsmath}
\usepackage{amsfonts}
\usepackage{amssymb}
%\author{}
%\title{}
%\setbeamercovered{transparent} 
%\setbeamertemplate{navigation symbols}{} 
%\logo{} 
%\institute{} 
%\date{} 
%\subject{} 
\begin{document}

%\begin{frame}
%\titlepage
%\end{frame}

%\begin{frame}
%\tableofcontents
%\end{frame}

\begin{frame}{Gesund - Krank - Tod}

Das Ebola-Fieber wird durch einen Virus ausgelöst. Zunächst haben Patienten sehr unspezifische Symptome, wie Fieber und Durchfall. Jedoch werden feine Blutgefäße durch die Infektion zerstört, wodurch es zu Nieren- und Leberschäden kommen kann, sowie inneren Blutungen, die sogar zu Kreislaufversagen und Tod führen können.\\
Jeder, der mit Ebola infiziert wurde, ist etwa 10 Tage krank. In dieser Zeit kann er etwa 2,18 weitere Personen anstecken. Von den Infizierten sterben etwa 50\% an der Krankheit, die andere Hälfte wird wieder gesund und kann erneut angesteckt werden.\\

Es soll nun ein Modell zur Simulation von Epidemien, in Abhängigkeit der Zeit, erarbeitet werden.\\
\end{frame}

\begin{frame}
Formuliere Rechenregeln, die dir helfen, die Anzahl der Gesunden, Kranken und Toten nach 10$\cdot$t Tagen abzulesen. Vergleiche deine Regeln, mit denen deines Nachbarn.\\

Versucht mit Hilfe eurer Rechenregeln und einem Tabellenkalkulationsprogramm den Verlauf der Ebola-Epidemie in Guinea, Liberia, Sierra Leone und Nigeria für ein Jahr (360 Tage) zu simulieren. In diesen Ländern leben 11,5 Millionen Menschen und es gibt 4108 bestätigte Fälle von Ebola-Fieber.\\ 
\end{frame}
\begin{frame}{Gesund - Krank - Tod - Immun - Hospitalisiert}

Wir nehmen an, dass Personen die einmal erkrankt sind, nicht mehr erkranken. Zusätzlich lassen sich 70\% der Infizierten in einem Krankenhaus behandeln, wodurch die Überlebenswahrscheinlichkeit auf 60\% steigt.\\

Erweitert euer Liste der Rechenregeln mit Hilfe der neuen Informationen und setzt auch diese mit einem Tabellenkalkulationsprogramm um.

\end{frame}
\begin{frame} {Zusatz}

Auch Tote sind noch ansteckend. Das Virus befindet sich in Körperflüssigkeiten und kann durch die Schleimhäute gesunder Personen aufgenommen werden. Da in Afrika bestimmte Beerdigungsrituale durchgeführt werden, ist es wahrscheinlich, sich bei diesen Gelegenheiten anzustecken. Wir nehmen eine Wahrscheinlichkeit von 20\% an. Wir gehen jedoch davon aus, dass diejenigen, die im Krankenhaus gestorben sind, ohne Ansteckungsgefahr für andere beerdigt werden.

\end{frame}

\begin{frame}
Erweitert eure Rechenregeln mit Hilfe der neuen Informationen und setzt auch diese mit einem Tabellenkalkulationsprogramm um. 
\end{frame}
\end{document}