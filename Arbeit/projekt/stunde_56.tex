\subsection{Projektstunde 5 \& 6}\steffen
\subsubsection{Bemerkungen zur Lerngruppe}
Ergänzend zu den Einschätzungen in Abschnitt \ref{ssec:project:34:lerngruppe} zeigt sich die Lerngruppe als durchaus motiviert. Die Schülerinnen und Schüler, die in der 1. und 2. Stunde wegen innerschulischer Aktivitäten nicht am Unterricht teilgenommen haben, gliedern sich gut in das Leistungsspektrum der Klasse ein. Es besteht weiterhin das für Informatikkurse charakteristische ``Leistungstal'' in der Mitte des Leistungsspektrums. Da es sich um eine Hochbegabtenklasse handelt, entspricht das untere Ende des Leistungsspektrums etwa einer durchschnittlichen Leistung in Regelschulen. Die Schülerinnen und Schüler sind interessiert am Kontext der Reihe, fordern aber auch Beschäftigung ein. Die Reaktion auf fragend-entwickelnden Unterricht lässt sich als zäh beschreiben. 

Das Vorwissen in der Klasse ist stark heterogen. Obwohl viele der Schüler ein Zertifikat über den Einsatz von Excel verfügen, ist der aktive Einsatz aber bei den meisten einige Jahre her. Dementsprechend muss beim Einsatz von Tabellenkalkulation entsprechend unterstützt werden. Informatische Grundbildung ist aber breit in der Klasse vorhanden. Begriffe wie ``Variable'' und ``Anweisung'' werden verstanden. 

Schüler R\footnote{Name wegen Datenschutz anonymisiert} hebt sich weiterhin von der restlichen Klasse ab im Hinblick auf die Leistung deutlich nach oben ab. Bezüglich der Sozialkompetenz zeigt sich aber Nachholbedarf. R fordert auch explizit mehr Handlungsspielräume im Unterricht.

Eine Gruppe von fünf Schülerinnen erfordert eine etwas stärkere Lehrerpräsenz während der Bearbeitung von Aufgaben, da sonst deren Aufmerksamkeit sich anderen, unterrichtsirrelevanten Themen zuwendet. 

Eine weitere Gruppe von drei Schülern benötigen ebenfalls mehr Lehrerpräsenz, da diese Gruppe zu alternativen Lösungsansätzen neigt, die jedoch in den vergangen Stunden nicht immer zu einer Lösung führt.
\subsubsection{Methodische Überlegungen}
Aus den Beobachtungen innerhalb der Vorstunde ging hervor, dass es in der Klasse noch Probleme mit der Interpretation des Netzwerkmodells gibt. Vor allem, dass die Anzahl der Individuen innerhalb des Modells konstant bleibt. Es wurde häufig beobachtet, dass die Schüler ausgehende Kanten nicht beachten. Da dieses Verständnis aber essentiell für den weiteren Verlauf der Reihe ist, muss hier entsprechend korrigierend gewirkt werden. Eine Möglichkeit wäre es, dies mit einem Lehrvortrag zu tun, in dem diese Zusammenhänge explizit erklärt werden. Die Lerngruppe neigt aber dazu, bei Vorträgen sehr schnell das Interesse und damit auch die Aufmerksamkeit zu verlieren. Aus diesem Grund sollte auf Vorträge in dieser Klasse verzichtet werden. Organischer und Schülerzentrierter ist dagegen der folgende Ansatz: Den Schülern wird das Netzwerksmodell präsentiert. Anschließend sollen sie zuerst den Weg eines Individuums durch den Graph beschreiben. Anschließend werden die Iterationsvorschriften in nicht-zusammengefasster Form präsentiert. Die Schüler sollen sich nun gegenseitig mittels \emph{Think-Pair-Share} die Vorschrift für $K(n)$ erklären. 

Nachdem das Modell für alle einheitlich festgesetzt wurde, sollen nun Maßnahmen zur Eindämmung einer Krankheit modelliert werden. Der bisherige Ansatz wurde von der Klasse als zu kleinschrittig empfunden. Um weiterhin die Motivation der Klasse zu erhalten, wird der Arbeitsauftrag wesentlich freier als bisher gestaltet. Informationen, die die Schüler in eine gewisse Richtung führen sollen, werden nicht gegeben. Der Auftrag beinhaltet die Modellierung der Maßnahmen sowie deren Integration in das bestehende Modell und der Auswertung mit Excel. Gefordert werden mindestens zwei Maßnahmen. Durch die Formulierung ``mindestens'' können die Leistungsspitzen in der Klasse abgefangen werden. Um schwächere Schüler nicht zu verlieren, wird der Auftrag als Gruppenarbeit gegeben. Eine Bearbeitung mit maximal drei Schülern erachte ich als praktikabel. Alternativ ließen sich die Maßnahmen auch fragend-entwickelnd modellieren. Die Lerngruppe scheint aber noch nicht genug mit einem schülerzentrierten Unterricht vertraut zu sein, um die Maßnahmen in einer Unterrichtsdiskussion flüssig erarbeiten zu können. Die Diskussion liefe Gefahr zu lehrerzentriert zu werden und nur kleine Teile der Lerngruppe zu aktivieren. Der gruppenbasierte Ansatz ist darum vorzuziehen.

Die Sicherung der Ergebnisse soll ebenfalls schülerzentriert sein. Jede Gruppe ihre Ergebnisse komplett vorstellen zu lassen, würde wahrscheinlich zu viel Redundanz führen. Dadurch würden große Teile der Klasse in dieser Zeit sich anderen Themen zuwenden. 

\begin{landscape}
\subsubsection{Verlaufsskizze}
\noindent
\begin{longtable}{|C{0.3\textwidth}|L{0.8\textwidth}|L{0.4\textwidth}|}
\hline
Phase & Arbeitsauftrag & Sozialform\\
\hline\hline
\endhead
\hline
\endfoot
Einstieg& Wdh: Zustandsmodell \& Iterationsvorschriften. ``Ein Individuum namens Alice ist gesund. Beschreibe anhand des Modells, was mit Alice in den nächsten Tagen passiert''

Anschließend:``Wieso sieht $K(n)$ so aus?'' in TPS& Unterrichtsgespräch\\\hline
Auftragsübergabe&``Welche Möglichkeiten gibt es, die Ausbreitung einer Krankheit zu verhindern oder zu verlangsamen? Modelliert mindestens 2 Maßnahmen und integriert diese in das bestehende Modell.''

Puffer: ``Integriert die Maßnahmen in Euer TK-Sheet''&Lehrervortrag\\\hline
Erarbeitung&Zeitansatz $\approx$ 20min&Einzel- oder Partnerarbeit\\\hline
Evaluation&Schüler präsentieren Ihre Ergebnisse. Bei Schülern, die die Zusatzaufgabe erfüllt haben, wird zudem gefragt ``Wie haben sich die Maßnahmen auf den Krankheitsverlauf ausgewirkt''&Schülerpräsentation\\\hline
\end{longtable}
\end{landscape}
\subsubsection{Materialien}
\subsubsection{Erwartungshorizont}
\subsubsection{Schülerprodukte}
\subsubsection{Reflexion der Stunde}