\subsection{Projektstunde 5 \& 6}\steffen
\subsubsection{Bemerkungen zur Lerngruppe}
Ergänzend zu den Einschätzungen in Abschnitt \ref{ssec:project:34:lerngruppe} zeigt sich die Lerngruppe als durchaus motiviert. Die Schülerinnen und Schüler, die in der 1. und 2. Stunde wegen innerschulischer Aktivitäten nicht am Unterricht teilgenommen haben, gliedern sich gut in das Leistungsspektrum der Klasse ein. Es besteht weiterhin das für Informatikkurse charakteristische ``Leistungstal'' in der Mitte des Leistungsspektrums. Da es sich um eine Hochbegabtenklasse handelt, entspricht das untere Ende des Leistungsspektrums etwa einer durchschnittlichen Leistung in Regelschulen. Die Schülerinnen und Schüler sind interessiert am Kontext der Reihe, fordern aber auch Beschäftigung ein. Die Reaktion auf fragend-entwickelnden Unterricht lässt sich als zäh beschreiben. 

Das Vorwissen in der Klasse ist stark heterogen. Obwohl viele der Schüler ein Zertifikat über den Einsatz von Excel verfügen, ist der aktive Einsatz aber bei den meisten einige Jahre her. Dementsprechend muss beim Einsatz von Tabellenkalkulation entsprechend unterstützt werden. Informatische Grundbildung ist aber breit in der Klasse vorhanden. Begriffe wie ``Variable'' und ``Anweisung'' werden verstanden. 

Schüler R\footnote{Name wegen Datenschutz anonymisiert} hebt sich weiterhin von der restlichen Klasse ab im Hinblick auf die Leistung deutlich nach oben ab. Bezüglich der Sozialkompetenz zeigt sich aber Nachholbedarf. R fordert auch explizit mehr Handlungsspielräume im Unterricht.

Eine Gruppe von fünf Schülerinnen erfordert eine etwas stärkere Lehrerpräsenz während der Bearbeitung von Aufgaben, da sonst deren Aufmerksamkeit sich anderen, unterrichtsirrelevanten Themen zuwendet. 

Eine weitere Gruppe von drei Schülern benötigen ebenfalls mehr Lehrerpräsenz, da diese Gruppe zu alternativen Lösungsansätzen neigt, die jedoch in den vergangen Stunden nicht immer zu einer Lösung führt.
\subsubsection{Methodische Überlegungen}
\begin{landscape}
\subsubsection{Verlaufsskizze}
\noindent
\begin{longtable}{|C{0.3\textwidth}|L{0.8\textwidth}|L{0.4\textwidth}|}
\hline
Phase & Arbeitsauftrag & Sozialform\\
\hline\hline
\endhead
\hline
\endfoot
Einstieg& Wdh: Zustandsmodell \& Iterationsvorschriften. ``Ein Individuum namens Alice ist gesund. Beschreibe anhand des Modells, was mit Alice in den nächsten Tagen passiert''& Unterrichtsgespräch\\\hline
Auftragsübergabe&``Welche Möglichkeiten gibt es, die Ausbreitung einer Krankheit zu verhindern oder zu verlangsamen? Modelliert mindestens 2 Maßnahmen und integriert diese in das bestehende Modell.''

Puffer: ``Integriert die Maßnahmen in Euer TK-Sheet''&Lehrervortrag\\\hline
Erarbeitung&Zeitansatz $\approx$ 20min&Einzel- oder Partnerarbeit\\\hline
Evaluation&Schüler präsentieren Ihre Ergebnisse. Bei Schülern, die die Zusatzaufgabe erfüllt haben, wird zudem gefragt ``Wie haben sich die Maßnahmen auf den Krankheitsverlauf ausgewirkt''&Schülerpräsentation\\\hline
\end{longtable}
\end{landscape}
\subsubsection{Materialien}
\subsubsection{Erwartungshorizont}
\subsubsection{Schülerprodukte}
\subsubsection{Reflexion der Stunde}