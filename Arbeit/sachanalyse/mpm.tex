\steffen
\begin{description}
	\item[\cite{Sattenspiel1995}] Erweiterung des SIR-Models um Reisen zwischen den Populationen. Individuen reisen zu einem Ziel und kehren zurück. Die Reisebereitschaft ist invariant unter Infektionen oder anderen Faktoren. Im Prinzip ein Zusammenschluss von Mobilitäts- und SIR-Modell
	\item[\cite{Capasso1978}] Behandelt die Effektivität von Schulschließungen während einer Grippewelle. Arbeitet mit Individuen, die sich zwischen verschiedenen Gruppen hin und her bewegen (explizit: Familie, Schule, Arbeit und Krankenhaus). Die Übertragungsraten sind abhängig von der Gruppe, in der sich ein Individuum befindet. Das Schließen von Schulen zeigt in der Simulation eindeutige Vorteile.
\end{description}

\begin{itemize}
	\item Aufbauend auf dem beschriebenen SIR Modell in \ref{ssec:spa}
	\item Verwendung eines Netzwerkes
	\item Jeder Knoten ist eine Zustandspopulation (Terminus festlegen). Jede Kante beschreibt einen Fluss von einem Knoten zu einem Anderen.
	\item eine SIR Population ist demnach ein Subnetz
	\item Knoten sind nie negativ. 
	\item diskrete Zeitschritte
\end{itemize}