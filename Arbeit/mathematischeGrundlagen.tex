\section{Mathematische Grundlagen}
\subsection{Single Population Ansätze}

Zunächst werden Single Population Ansätze untersucht, also Modelle zur Beschreibung einer Gesellschaft, beispielsweise eines Landes oder einer Stadt. Dazu werden üblicherweise SIR-Modelle genutzt.

\subsubsection{SIR Modell}

Das klassische SIR-Modell ist ein Ansatz über gewöhnliche Differenzialgleichungen. Dabei gilt, dass die gesamte Bevölkerungsanzahl sich nicht verändert.\\
Individuen der Bevölkerung befinden sich in einer von drei Gruppen.\\
Entweder ist das Individuum gesund und kann angesteckt werden (Gruppe S), es ist krank und kann Personen aus der Gruppe S infizieren (Gruppe I), oder es kann nicht mehr angesteckt werden (immun, tot oder unter Quarantäne) und befindet sich in der Gruppe R.\\
Sei nun $N$ die gesamte Bevölkerungsanzahl, dann gilt im klassischen Modell zu jedem Zeitpunkt $t$, $S(t)+I(t)+R(t) = N$. Zusätzlich wird die Annahme getroffen, dass sich zu Beginn einer Epidemie niemand in der Gruppe R befindet, also gilt: $S(0)+I(0)= S_0+I_0 = N$.\\
Die Ausbreitung der Infektion wird dann mit der folgenden Gruppe von Differenzialgleichungen beschrieben.
\begin{align}
\dfrac{dS}{dt} = -kSI\\
\dfrac{dI}{dt} = kSI- \gamma I\\
\dfrac{dR}{dt} = \gamma I
\end{align}

Dabei gilt: $S_0 > 0, I_0 > 0, R_0 = 0$.\\
$k > 0$ beschreibt die Infektionsrate und $\gamma > 0$ die Immunisierungsrate, wobei zu beachten ist, dass auch Tote als immun angenommen werden. \\
Der Faktor $SI$ beschreibt die Möglichkeiten, dass eine kranke auf eine gesunde Person trifft um diese anstecken zu können.\\  
Dieses Modell ist sehr simpel, hat jedoch einige schwächen im Bezug auf die Beobachtungen in der Realität.\\
So scheint es unwahrscheinlich, dass der Interaktionsterm linear ist. Viel wahrscheinlicher ist es, dass Aufgrund von Verhaltensänderungen eine Verringerung der Interaktionen eintritt, sobald die Epidemie bekannt ist.\\
Zudem ist es bei Beobachtungen über einen längeren Zeitraum durchaus wahrscheinlich, dass es zu Geburten und Todesfällen kommt.\\
Um diesen Überlegungen Rechnung zu tragen wurde das SIR-Modell bereits oft weiterentwickelt.


\subsubsection{Modifizierte SIR Modelle}
\subsubsection{IDEA Modell}
\subsection{Netzwerkansätze}