\section{Mathematische Grundlagen}
\subsection{Single Population Ansätze}\label{ssec:spa}
\ellen
Zunächst werden Single Population Ansätze untersucht. Dabei handelt es sich um Modelle zur Beschreibung einer Gesellschaft, beispielsweise eines Landes oder einer Stadt. Dazu werden üblicherweise SIR-Modelle genutzt.

\subsubsection{SIR Modell}
\ellen
Das klassische SIR-Modell ist ein Ansatz über gewöhnliche Differenzialgleichungen. Dabei gilt, dass die gesamte Bevölkerungsanzahl sich nicht verändert.\\
Individuen der Bevölkerung befinden sich in einer von drei Gruppen.\\
\begin{itemize}
\item Entweder ist das Individuum gesund und kann angesteckt werden (Gruppe S), 
\item es ist krank und kann Personen aus der Gruppe S infizieren (Gruppe I) 
\item oder es kann nicht mehr angesteckt werden (immun, tot oder unter Quarantäne) und befindet sich in der Gruppe R.
\end{itemize}
Sei nun $N$ die gesamte Bevölkerungsanzahl, dann gilt im klassischen Modell zu jedem Zeitpunkt $t$, $S(t)+I(t)+R(t) = N$. Zusätzlich wird die Annahme getroffen, dass sich zu Beginn einer Epidemie niemand in der Gruppe R befindet, weswegen gilt: $S(0)+I(0)= S_0+I_0 = N$.\\
Die Ausbreitung der Infektion wird dann mit den folgenden Differenzialgleichungen beschrieben.
\begin{align}
\dfrac{dS}{dt} &= -kSI\\
\dfrac{dI}{dt} &= kSI- \gamma I\\
\dfrac{dR}{dt} &= \gamma I
\end{align}

Dabei gilt: $S_0 > 0 \& I_0 > 0 und R_0 = 0$.\\
$k > 0$ beschreibt die Infektionsrate und $\gamma > 0$ die Immunisierungsrate, wobei zu beachten ist, dass auch Tote als immun angenommen werden. \\
Der Faktor $SI$ beschreibt die Möglichkeiten, dass eine kranke auf eine gesunde Person trifft um diese anstecken zu können(\citep{Capasso1978}).\\  
Dieses Modell ist sehr simpel, hat jedoch einige Schwächen im Bezug auf die Beobachtungen in der Realität.\\
So scheint es unwahrscheinlich, dass der Interaktionsterm linear ist. Viel wahrscheinlicher ist es, dass Aufgrund von Verhaltensänderungen eine Verringerung der Interaktionen eintritt, sobald die Epidemie bekannt wird.\\
Zudem ist es bei Beobachtungen über einen längeren Zeitraum durchaus wahrscheinlich, dass es zu Geburten und Todesfällen kommt.\\
Um diesen Überlegungen Rechnung zu tragen, wurde das SIR-Modell bereits oft weiterentwickelt.\\
Eine Auswahl dieser Weiterentwicklungen wird später beschrieben und untersucht. Zunächst wird jedoch das klassische SIR-Modell diskretisiert. Dies ist für die vorliegende Arbeit notwendig, da eine Umsetzung in der Schule durchgeführt werden soll und Differentialgleichungen in der Schule nicht zwingend behandelt werden.\\
 Da rekursive Folgen bereits früh von SuS untersucht werden, bietet es sich an, die Differentialgleichungen in rekursive Funktionen umzuwandeln.\\
  Dann kann man die Gleichungen folgendermaßen beschreiben:
  \begin{align}
  S_n &= S_{n-1}- kS_{n-1}I_{n-1}\\
  I_n &= I_{n-1}+ kS_{n-1}I_{n-1}- \gamma I_{n-1}\\
  R_n &= R_{n-1}+ \gamma I_{n-1}
  \end{align}


\subsubsection{Modifizierte SIR Modelle}
\ellen
Eine einfache Möglichkeit zur Modifikation des SIR-Modells ist es, zusätzliche Zustände einzuführen. \cite{Rivers2014} verfolgen einen solchen Ansatz und erweitern das SIR-Modell zu einem SEIHFR-Modell. 
Genau wie im grundlegenden Modell gibt es die Ansteckbaren (S), Infektiösen (I) und die, die nicht mehr angesteckt werden und auch nicht mehr anstecken (R). Zusätzlich gibt es jedoch jene, die Kontakt zu Kranken hatten, jedoch noch nicht infektiös sind (E), sowie jene die in Krankenhäusern untergebracht sind (H) und jene, die sterben und beerdigt werden (F).
Durch die Einführung zusätzlicher Zustände können genauere Aussagen getroffen werden, da beispielsweise die Klasse R zuvor ungenügend Informationen zur Verfügung stellte. Durch das SEIHFR-Modell ist es möglich zu unterscheiden, wie viele immunisiert wurden und wie viele gestorben sind. Dies ist im weiteren Verlauf wichtig, wenn man Methoden zur Eindämmung und Bekämpfung von Infektionskrankheiten untersuchen möchte. Gleichzeitig wird das Problem jedoch komplizierter durch die Erweiterung des Modells, da die einzelnen Zustände nun komplizierter miteinander interagieren.\\
So wird ein Teil von I in Krankenhäuser eingeliefert (H), während andere sterben (F) und wieder andere genesen und immunisiert werden (R). Ebenso stirb ein Teil derjenigen, die sich in Krankenhäusern befinden, während andere genesen und in den Zustand R übergehen.\\
Da es sich hier jedoch um Übergänge handelt, die durchaus nachvollziehbar und für Schülerinnen und Schüler begreiflich darstellbar sind, ist der Gewinn durch die differenziertere Betrachtung der Klasse R stärker zu bewerten, als die Kosten die durch den höheren Aufwand entstehen.\\
\cite{Roberts2006} untersuchten 2007 ebenfalls die Möglichkeit Epidemien zu kontrollieren und dadurch die Auswirkungen möglichst gering zu halten. Dafür unterteilten sie die Klasse der Infektiösen nach dem Ort, an dem sie sich infiziert haben (bei einem Familienmitglied, in der Schule, auf der Arbeit oder bei der Gesellschaft). Dadurch entsteht die Möglichkeit, die Auswirkungen von der Schließung bestimmter Einrichtungen, wie beispielsweise Schulen, bezüglich der Ausbreitung der Krankheit zu untersuchen. Auf diese Untersuchung wird jedoch in dieser Arbeit verzichtet, da das Modell sonst zu komplex für die Behandlung im Unterricht wird. \\
\subsubsection{IDEA Modell}
\ellen
Auch das IDEA-Modell (Incidence decay with exponential adjustment) versucht die Ausbreitung von Infektionskrankheiten zu beschreiben. Dieses Modell benutzt keine Differentialgleichungen, sondern beschränkt sich auf eine Funktion:\\
\begin{align}
I_t &=\left(\dfrac{R_0}{(1+d)^t}\right)^t \label{eq:IDEA}
\end{align}
Dabei beschreibt $I_t$ die Menge der Krankheitsfälle zum Zeitpunkt t, $R_0$ die Grundreproduktionsrate und $d$ ist ein Kontrollparameter, der dafür sorgt, dass die Anzahl der Krankheitsfälle nicht zu stark ansteigt (vgl. \cite{Fisman2014}). 
\subsubsection{Wahl eines Modells}
\ellen
Das IDEA-Modell kann einfach im Unterricht eingebunden werden, da keine unbekannten Größen oder Operationen für die Schülerinnen und Schüler auftreten. Da jedoch lediglich die Klasse der Infizierten beschrieben wird, ist sie für die weitere Arbeit kaum zu nutzen, da die Untersuchung der Klasse R nicht möglich ist.\\
 Auch das klassische SIR-Modell hält nicht genügend Möglichkeiten bereit, die Klasse R zu untersuchen.\\
  Das SEIHFR-Modell bietet diese Möglichkeiten, jedoch wird dieses, um die Auswertung zu vereinfachen, um die Klasse D erweitert. Diese Klasse beschreibt jene, die durch die Krankheit gestorben sind. 
  \begin{figure}
\begin{center}
\begin{tikzpicture}[->,>=stealth',shorten >=1pt,auto, node distance=3cm]
	\node[state] (S)				{S};
	\node[state] (E) [right =4cm of S]	{E};
	\node[state] (I) [right of=E]	{I};
	\node[state] (H) [right of=I]	{H};
	\node[state] (F)[above of=H]	{F};
	\node[state] (D)[right of=F]	{D};
	\node[state] (R)[below right of=H]	{R};
	\path 	(S)	edge node {$\alpha \cdot (I+E)+ \omega \cdot F$}(E)
			(E)	edge node {$1 \cdot E$}							(I)
			(I)	edge node {$\beta \cdot I$}						(F)
			(I)	edge node {$\gamma_\cdot I$}					(H)
			(I)	edge node {$\delta \cdot I$}					(R)
			(F)	edge node {$1 \cdot F $}						(D)
			(H)	edge node {$\epsilon \cdot H$}					(D)
			(H) edge node {$(1-\epsilon) \cdot H $}				(R);
	
\end{tikzpicture}
\end{center}
\caption{SEIHFRD-Modell}\label{fig:ssec:model}
\end{figure}
In Abbildung \ref{fig:ssec:model} entspricht $\alpha$ der Reproduktionsrate, also der Anzahl der Gesunden, die durch einen Infektiösen angesteckt werden. Forscher der Eidgenössischen Technischen Hochschule Zürich (ETH Zürich) stellten fest, dass diese Rate bei Ebola etwa bei 2,18 liegt (\cite{Stadler2014}).\\
Es kann davon ausgegangen werden, dass alle die ansteckend sind auch Symptome ausbilden.\\
Mit Hilfe der Reproduktionsrate und dem Wissen über die Menge an Personen, die im Krankenhaus behandelt wurden, konnten die Forscher ebenso die Dunkelziffer abschätzen. Diese liegt für Ebola bei 30\%.\\
Es starben etwa 50\% der Infizierten bei der Epidemie in Sierra Leone und Guinea. Man kann davon ausgehen, dass diese Rate bei jenen die sich nicht in Behandlung befinden etwas höher liegt, während sie bei denen im Krankenhaus niedriger liegt.\\

\cite{Rivers2014} geben in ihrem Paper an, dass die Sterblichkeitsrate durch Krankenhausaufenthalte nicht verändert wird. Eine Studie von Ärzte ohne Grenzen kommt jedoch zu dem Schluss, dass ein Malariamedikament die Sterblichkeitsrate um 31\% senkt (\cite{AerzteGrenzen2016}).
Daher setzen wir die Sterblichkeitsrate im Krankenhaus $\epsilon = 0.4$. Dadurch ist die Überlebenswahrscheinlichkeit gegenüber dem Ausbleiben einer Behandlung leicht erhöht. Dies wirkt mit dem Alltagsverständnis vereinbar, jedoch ist die Verbesserung von 31\% zu hoch, da die Verfügbarkeit des Malariamedikaments nicht in allen Krankenhäusern und für alle Patienten gewährleistet werden kann.

Damit kann man annehmen, dass ohne Behandlung sechs von zehn Patienten sterben. Das bedeutet für das obige Modell, dass $\beta$ 0,18 entspricht, während $\delta$ 0,12 entspricht, denn $\beta + \delta = 0,3$. Zusätzlich gilt, dass $\beta + \delta + \gamma = 1$. \\
\subsection{Erweiterung auf mehrere miteinander interagierende Populationen}\label{ssec:multiPop}
\steffen
\steffen
\begin{description}
	\item[\cite{Sattenspiel1995}] Erweiterung des SIR-Models um Reisen zwischen den Populationen. Individuen reisen zu einem Ziel und kehren zurück. Die Reisebereitschaft ist invariant unter Infektionen oder anderen Faktoren. Im Prinzip ein Zusammenschluss von Mobilitäts- und SIR-Modell
	\item[\cite{Capasso1978}] Behandelt die Effektivität von Schulschließungen während einer Grippewelle. Arbeitet mit Individuen, die sich zwischen verschiedenen Gruppen hin und her bewegen (explizit: Familie, Schule, Arbeit und Krankenhaus). Die Übertragungsraten sind abhängig von der Gruppe, in der sich ein Individuum befindet. Das Schließen von Schulen zeigt in der Simulation eindeutige Vorteile.
\end{description}

\begin{itemize}
	\item Aufbauend auf dem beschriebenen SIR Modell in \ref{ssec:spa}
	\item Verwendung eines Netzwerkes
	\item Jeder Knoten ist eine Zustandspopulation (Terminus festlegen). Jede Kante beschreibt einen Fluss von einem Knoten zu einem Anderen.
	\item eine SIR Population ist demnach ein Subnetz
	\item Knoten sind nie negativ. 
	\item diskrete Zeitschritte
\end{itemize}