%-----Papier-----%
\documentclass[12pt, a4paper]{article}
\usepackage[a4paper,lmargin={4cm},rmargin={2cm},tmargin={2.5cm},bmargin = {2.5cm}]{geometry}

\usepackage[utf8]{inputenc} %Zeichenunterstützung für Umlaute u.ä.
%-----AMS-----%
\usepackage{amsmath} %Mathepakete
\usepackage{amsfonts}
\usepackage{amssymb}
%-----Layout-----%
\usepackage{ngerman} %Deutsches Layout
\usepackage[onehalfspacing]{setspace} %Zeilenabstand
\usepackage[colorlinks=true,urlcolor=blue, linkcolor=black, citecolor=black]{hyperref} %Links innerhalb des Dokuments, alle schwarz
\sloppy %Blocksatz wird erzwungen (sinnvoll, da Latex Probleme hat, deutsche Wörter zu trennen
%-----Pakete-----%
\usepackage{hyperref} % Hyperlinks %http://en.wikibooks.org/wiki/LaTeX/Labels_and_Cross-referencing
\usepackage{graphicx} %Einbinden von Bildern
\usepackage{float} %Erlaubt figures mit option H fest im text zu verankern
\usepackage{caption} %Erlaubt unnummerierte figure-captions
\usepackage{listings} %Erlaubt das (automatische) Einbinden von SourceCode http://en.wikibooks.org/wiki/LaTeX/Source_Code_Listings
\usepackage{tikz} %Zum Zeichnen von Diagrammen http://csweb.ucc.ie/~dongen/LAF/TikZ.pdf
\usepackage{tabularx}%Fuer eigene Tabulararten
\usepackage{longtable}

%-----Literatur-----%
\usepackage{res/bib/natbib} 
\bibliographystyle{res/bib/natdin}


\usepackage{pgffor}


\usetikzlibrary{arrows,automata,positioning}